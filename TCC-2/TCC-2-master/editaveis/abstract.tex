\begin{resumo}[Abstract]
 \begin{otherlanguage*}{english}

Any content to be made available and shared on the web, requires special attention from programming languages to be used, from the simplest to the most complex, depending on the content to be made available. The knowledge required usually has a certain complexity that may not be common to lay users. This situation can be mitigated by the use of systems of content management (CMS - Content Management System), which, in turn are platforms that make the union between various mechanisms that allow you to create and publish content in real time. But the choice of a particular CMS can be something not so easy, why such systems allow to be used in a variety of contexts and situations. Thus, the choice of a CMS can be facilitated with a clear method to be followed by those who will take the decision of which CMS have to be use in their specific contexts. The method proposed in this work takes into account the main features found in type systems CMS, organizing a procedure that will lead the user to a choice that is advantageous, because these systems can provide the necessary support for the construction of: social networks, e-Commerce sites, electronic journals, among other possibilities.

   \vspace{\onelineskip}
   \noindent 
   \textbf{Keywords}: CMS. Method. Web Development.  Software Quality. SQuaRE. Features. 
 \end{otherlanguage*}
\end{resumo}
