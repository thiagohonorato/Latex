\begin{resumo}

Qualquer conteúdo para ser divulgado e compartilhado na web, requer atenção especial das linguagens de programação a serem utilizadas, desde as mais simples até as mais complexas, dependendo do conteúdo a ser disponibilizado. O conhecimento necessário costuma ter uma determinada complexidade que pode não ser comum a usuários leigos. Esta situação pode ser amenizada pelo uso de sistemas de gestão de conteúdo (CMS – Content Management System) que, por sua vez, são plataformas que fazem a união entre diversos mecanismos que permitem criar e publicar conteúdo em tempo real. Porém a escolha de um determinado CMS pode ser algo não muito fácil,  pois tais sistemas permitem serem utilizados em diversos contextos e situações. Assim sendo, a escolha de um CMS pode ser facilitada com um método claro a ser seguido por aqueles que necessitarem tomar a decisão de qual CMS usar no seu contexto específico. O método proposto neste trabalho leva em consideração as principais características encontradas em sistemas do tipo CMS, organizando um procedimento que conduza o usuário a uma escolha que lhe seja vantajosa, pois esses sistemas podem fornecer o apoio necessário para a construção de redes sociais, sites de e-Commerce, revistas eletrônicas, entre outras possibilidades.  
 
 \vspace{\onelineskip}
 \noindent
 \textbf{Palavras-chaves}: CMS. Método. Desenvolvimento Web. Qualidade de Software. SQuaRE. Características.  
\end{resumo}